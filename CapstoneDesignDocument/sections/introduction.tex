\chapter{Introduction}


\section{Problem Statement} 
%-------------------------------------------%

Can we create a fair and balanced artificial intelligence that plays video games almost like a human? How can we apply artificial intelligence in entertainment?

%-------------------------------------------%



\section{Purpose Statement} 
%-------------------------------------------%

We shall create an artificial intelligence that uses the same information available to players in order to make decisions. We shall also closely train the Agent to reproduce human-like on a reliable scale. 


%-------------------------------------------%



\section{Context} 
%-------------------------------------------%

Interest in the field of artificial intelligence has been present since nearly the beginning of computing. However, over the past few years interest in AI has been growing. With the recent popularity of machine learning techniques like neural networks, many researchers have made huge advancements in the field. Companies such Google are constantly appearing in mainstream media for their groundbreaking projects. Google's AlphaGo is an amazing example of the work that can be done; AlphaGo is an AI was able to beat the world's best Go player, and it continues to improve its game.

Other groups are contributing to the advancement of AI, although they are not as easily recognizable. A research platform called ViZDoom is one such example that closely relates to our work. ViZDoom is an Open Source platform that allows users to create agents in the videogame Doom, these game agents are doing what any human player would: trying to maximize their score. However, unlike more conventional game AI, they are playing using only the visual buffer. This approach more closely resembles how humans play the game, and was also a large inspiration to this project. Yet, ViZDoom is still primarily a research platform.

Finding cutting-edge AI techniques in the game industry is quite hard. Many developers find these techniques to be both impractical and excessive. These implementations use up sought after resources. Why should they create more work for themselves when many of the problems that we are looking to solve can be reduced to smaller problems with nice well defined algorithms?

With this information, one may think that almost no one uses advanced A.I. in games, and that is primarily the case. However, some companies are implementing these advanced techniques. Instead of using AI to create a diverse experience, they use it to push sales for specific features. This is the case with Activision's recently acquired patent, which defines a method for manipulating players into engaging with microtransactions.

%-------------------------------------------%


\section{Significance of Project} 
%-------------------------------------------%

This project's significance comes from the current scarcity of A.I. projects in entertainment. A.I. is one of the quickest growing fields right now, we think that it is important to create a precedent and show both developers and users what these advanced techniques can be used for. If these techniques are continued to be used to take advantage of users, it is possible that A.I. will be "stuck" in this state.

%-------------------------------------------%
