\chapter{Methods (5-7 pages)}


\section{Design}
%-------------------------------------------%

Design details of your project. This section can include various descriptions, figures, diagrams, wireframes, etc that are driving your project implementation.

\begin{figure}
	\caption{"Data flow within our software stack"}
	\centering
		\includegraphics[scale = 0.75]{dataflow.png} \\
\end{figure}

%-------------------------------------------%



\section{Environment}
%-------------------------------------------%

We have opted to use our own personal computers running 64-bit Windows 10 with Ubuntu subsystems as both our platform and development environment. There are several reasons to justify this decision. First, we wanted to ensure that we would have the best possible experience when dealing with GPU computation, and NVIDIA simply has a longer and better track record for supporting Windows than it does for Linux. Second, we wanted to avoid the potential pitfall of our platform restricting us to a smaller library of target games. Our two top candidates were {\it Quake} and {\it Rivals of Aether} because both of these games support replays, or demos; however, only one of them natively supports Linux. Last, and with regards to using our own computers for both development and testing, we made the decision to abstain from distributed or cloud computing chiefly to save costs, but also as an aesthetic choice based in the desire to see what our computers are capable of.

Despite all of the above justifications for using Windows, it remains to be said that our project cannot live without Linux. For reasons discussed in the next section, we use a Redis server which runs on an Ubuntu subsystem. The solution may not sound elegant, but it was certainly simple to set up.

%-------------------------------------------%



\section{Frameworks}
%-------------------------------------------%

Our project uses a software stack comprised of the following modules and libraries: SerpentAI, Keras, TensorFlow-GPU, and the Anaconda distribution for Python 3.

We chose to use Python 3 as our primary programming language for its flexibility, intuitive syntax, and data processing capabilities. Python is also helpfully compatible with all of the other software listed above. Furthermore, the Anaconda distribution provides us with a large assortment of Python modules, some of which SerpentAI cites as dependencies. This allows us to more easily work from within a Windows environment, which in turn gives us the most straightforward and stable access to GPU computation via NVIDIA proprietary drivers.

SerpentAI serves as a bridge between our artificial intelligence and its testing environment by providing a simple interface for retrieving frame buffer data from and sending control input to virtually any game runtime. This is achieved by writing a game agent program. Our game agent utilizes TensorFlow, a popular machine-learning library. However, does not access TensorFlow directly. Rather, it uses Keras, a high-level machine-learning library designed to serve as an abstraction layer over foundational libraries like TensorFlow. The low-level library still performs all of the heavy-lifting; Keras simply removes some of the barriers to entry.

%-------------------------------------------%



\section{Algorithms}
%-------------------------------------------%

Descriptions and pseudocode of important algorithms in use, relevant, or to be implemented as part of your project.

%-------------------------------------------%



\section{Analytical Methods}
%-------------------------------------------%

Descriptions and processes of any analytical methods used as part of your capstone project. This section is most applicable for projects with a data analysis focus or component. Quantitative and qualitative methods.

%-------------------------------------------%



\section{Features}
%-------------------------------------------%

A brief overview of primary features you plan to implement.

%-------------------------------------------%



\section{Test Plan}
%-------------------------------------------%

A brief overview of what and how you plan to test your project codebase.

%-------------------------------------------%



\section{Criteria and Constraints}
%-------------------------------------------%

Project development must meet desired needs within realistic constraints. Many times, constraints are interrelated. Cover the appropriate criteria and constraints related to your capstone. Some example criteria and constraints are health and safety, environmental, political, social, manufacturability, sustainability, economical, and ethical.

%-------------------------------------------%